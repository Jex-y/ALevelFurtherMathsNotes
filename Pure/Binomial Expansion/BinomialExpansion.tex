\documentclass[a4paper,12pt,fleqn]{article}
\usepackage{amssymb}
\usepackage[fleqn]{amsmath}
\setlength{\mathindent}{1cm}
\begin{document}

\title{Binomial Expansion}	
\author{Edward Jex}
\maketitle

Binomial Expansion is a way of easily expanding brackets with two terms to a power. The expansion is given by the general formula: 
\begin{equation*}
(x + y)^n = \sum_{r=0}^{n} ^nC_r \cdot x^{n-r} \cdot y^r
\end{equation*}
Note that since pascals triangle is symmetrical it doesn't matter which term way around the powers are. 
where $\binom{n}{c}$ is equivalent to $^nC_r$ and defined as:
\begin{equation*}
\binom{n}{c} = \frac{n!}{(n-r)! \cdot r!}
\end{equation*}
Example 1:
\begin{align*}
(2 + x)^4  & =
  (^4C_0 \cdot 2^4 \cdot x^0) 
+ (^4C_1 \cdot 2^3 \cdot x^1) 
+ (^4C_2 \cdot 2^2 \cdot x^2) 
+ (^4C_3 \cdot 2^1 \cdot x^3) 
+ (^4C_4 \cdot 2^0 \cdot x^4)&\\
& =
  (1 \cdot 16 \cdot 1) 
+ (4 \cdot 8 \cdot x) 
+ (6 \cdot 4 \cdot x^2) 
+ (4 \cdot 2 \cdot x^3)
+ (1 \cdot 1 \cdot x^4)&\\
& = x^4 + 8x^3 + 24x^2 + 32x + 16&\\
\end{align*}
\\
Example 2:
Find the $x^3$ coefficient in $(3 + 2x)^5$:
\\
let n = 5, r = 3
\begin{align*}
3^{rd}\; & = ^5C_3 \cdot 3^{5-3} \cdot (2x)^3 &\\
 & = 10 \cdot 3^2 \cdot 8x^3 &\\
 & = 720x^3 &\\
\therefore coefficent = 720
\end{align*}
\\
Example 3:
Find the constant term in $(x^2 + \frac{1}{x})^9$:\\
For the two terms to cancel out, the power of $\frac{1}{x}$ must be twice the power of $x^2$ so that when they are multiplied together they produce $x^0 = 1$, meaning the term is constant or independent $\implies$ We want the term where $x^2$ is to the power of 3 and $\frac{1}{x}$ is to the power of 6 $\implies$ constant term is:
\begin{align*}
t & = \: ^9C_3 \cdot (x^2)^3 \cdot \frac{1}{x}^6\\
  & = 84 \cdot x^6 \cdot x^{-6}\\
  & = 84
\end{align*}
\end{document}