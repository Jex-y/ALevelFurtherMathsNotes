\documentclass[a4paper,12pt]{article}
\usepackage[fleqn]{amsmath}
\usepackage{amssymb}
\begin{document}

\title{Moments}	
\author{Edward Jex}
\maketitle
A moment is the product of each force and its distance from a fixed point, the pivot. It models the effect of a force on an extended object. A rigid body is an object or body that is assumed to have a size and shape and not to be deformed when forces act upon it. Equilibrium is where the sum of all the moments is 0 as well as the sum of the forces so the motion is uniform. Levers can be used to lift heavy objects using a relativity small force, based on moments. \\

The moment can be found by $\vec{M} = \vec{F} \times \vec{X}$ Where F is the force and X is the displacement vector from the pivot. \\

In most questions, moments can be treated as scalars as it makes it much easier to solve. In this case $|\vec{M}| = |\vec{F}|.d$
\end{document}