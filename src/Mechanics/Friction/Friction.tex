\documentclass[a4paper,12pt]{article}
\usepackage[fleqn]{amsmath}
\usepackage{amssymb}
\begin{document}

\title{Friction}	
\author{Edward Jex}
\maketitle
\section*{Types of friction}
There are two types of friction: surface friction and friction in a fluid.
\subsection*{Friction in Fluids}
Depends on:
\begin{itemize}
	\item Materials in contact
	\item Viscosity of the fluid
	\item Shape and cross-sectional area of the object
	\item Velocity
\end{itemize}
\subsection*{Surface Friction} 
Surface friction is the friction between two solid surfaces. Kinetic friction is where the surfaces have a relative motion between them and static is where they are still. Static friction resist the surface starting to move and is often stronger than kinetic friction. \\
We model static friction simply by using a coefficient, $\mu$. This is called the coefficient of friction. 
$F_r = \mu F_n$ \\
Where Friction $F_r$ is the force due to friction and $F_n$ is the normal reaction force of the two surfaces. 

\end{document}