\documentclass[class=article, crop=false]{standalone}
\usepackage[fleqn]{amsmath}
\usepackage{amssymb}
\begin{document}

Sometimes it is more useful to find the probability of an event A given than an event B has occurred. We write this $p(A \mid B)$. \\
The conditional probability is defined as: \\
$p(A \mid B) = \frac{p(A \cap B)}{p(B)}$ \\
If events A and B are independent $p(A \mid B) = p(A)$ \\
Using tables is one of the easiest ways to work out the intersection.
\end{document}