\documentclass[a4paper,12pt]{article}
\usepackage[fleqn]{amsmath}
\usepackage{amssymb}
\begin{document}

\title{Probability}	
\author{Edward Jex}
\maketitle
\section*{Estimating Probability}
There are two ways to estimate probability:
\begin{itemize}
	\item Experiment
	\item Theoretical
\end{itemize}
\subsection*{Experimental}
$p = \frac{n_{\text{events}}}{n_{\text{trials}}}$ \\
Requires data to be collected. \\
\subsection*{Theoretical}
$p = \frac{n_{\text{ways}}}{n_{\text{outcomes}}}$ \\
For example, the probability of getting a 6 on a dice is $\frac{1}{6}$\\

\section*{Modelling Probability}
If A is impossible, $P(A) = 0$, if A is certain, $P(A) = 1$ \\
$P(A') = 1 - P(A)$ \\
If events A and B are  $P(A \cap B) = P(A) \times P(B)$ 
\subsection*{Mutually exclusive}
Two events are mutually exclusive events if they cannot both happen. \\
$P(A \cup B) = P(A) + P(B)$ \\

If events are not mutually exclusive $P(A \cup B) = P(A) + P(B) - P(A \cap B)$ \\
We can test for independence as if two events are independent $P(A \cap B) = P(A) \times P(B)$

\section*{Discrete Random Variables}
A model is a discrete random variable $X$ if it is:
\begin{itemize}
	\item Discrete
	\item The actual values of the outcome of the variable can only be predicted with a given probability.
\end{itemize}
Discrete Random Variables may have a finite or countably infinite number of outcomes.\\
\subsection*{Notation}
The particular values our DRV can take are denoted by r, this $P(X = r)$ means the probability that the DRV X has the outcome r. \\
The sum of these probabilities equal 1. Formally, $\sum_{r=1}^n P(X=r) = p_1 + p_2 + \dots + p_n = 1$\\

\section*{Expectation}
The most useful measure of central tendency is usually the mean (or the expectation). We can apply a similar idea for DRV. We define the expectation as: \\
$E(x) = \sum r P(X=r)$ \\
Note we often use the Greek symbol $\mu$ to represent $E(x)$ as well.\\
$\bar{x}$ is the mean when it is a sample.
$\mu$ is the mean when it is a population.
\end{document}