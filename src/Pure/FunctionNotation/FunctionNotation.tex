\documentclass[a4paper,12pt]{article}
\usepackage[fleqn]{amsmath}
\usepackage{amssymb}
\begin{document}

\title{Function Notation}	
\author{Edward Jex}
\maketitle
\section*{What is a function?}
\begin{align*}
f: A & \rightarrow B \\
x & \mapsto f(x)
\end{align*}
Where A is the domain (set of the possible inputs to the function), B is the range or co-domain (set of possible outputs from the function) and $x \mapsto f(x)$ is the mapping rule or operation. \\
The domain can be smaller than it needs to be and likewise the co-domain can be larger than it needs to be. 
\subsection*{Example 1:}
\begin{align*}
f: \mathbb{R} & \rightarrow \mathbb{R} \\
x & \mapsto x^2 \\\\
g: \mathbb{Z} & \rightarrow \mathbb{Z}^+ \\
x & \mapsto x^2 \\\\
h: \mathbb{R} \texttt{\char`\\} \{0\} & \rightarrow \mathbb{R} \\
x & \mapsto \frac{1}{x} \\\\
k: D & \rightarrow \mathbb{R} \\
\end{align*}
D doesn't include the number 0 in the set of reals.
\begin{align*}
x & \mapsto 4x^2 + 12x + 73 \\\\
D & = \{x \in \mathbb{R} \; \vert \mod{x,2} = 0 \}
\end{align*}
D is the set of even real numbers. 
\begin{align*}
l: [0,\infty) & \rightarrow \mathbb{R} \\
x & \mapsto \sqrt{x} 
\end{align*}
$[0,\infty)$ represents numbers larger than or equal 0 to smaller than infinity.
\section*{Intervals}
\begin{align*}
[a,b] & = \{ x \in \mathbb{R} \vert a \leqslant x \leqslant b \} \\
(a,b) & = \{ x \in \mathbb{R} \vert a < x < b \} \\
[a,b) & = \{ x \in \mathbb{R} \vert a \leqslant x < b \} \\
(a,b] & = \{ x \in \mathbb{R} \vert a < x \leqslant b \} \\
\end{align*}
\subsection*{Example 2:}
\begin{align*}
\text{Sine} : & \mathbb{R} \rightarrow [-1,1] \\
& x \mapsto \sin{x} \\
\text{Cosine}: & \mathbb{R} \rightarrow [-1,1] \\
& x \mapsto \cos{x} \\
\text{Tangent}: & \{ x \in \mathbb{R} \vert \mod{x + \frac{\pi}{2},\pi} \neq 0 \rightarrow [-1,1] \\
& x \mapsto \tan{x} \\
\end{align*}
\end{document}