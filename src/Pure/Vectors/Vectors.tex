\documentclass[a4paper,12pt]{article}
\usepackage[fleqn]{amsmath}
\usepackage{amssymb}
\usepackage{mathtools}
\usepackage{graphicx}
\graphicspath{ {./images/} }
\DeclarePairedDelimiter\abs{\lvert}{\rvert}%
\DeclarePairedDelimiter\norm{\lVert}{\rVert}%

\makeatletter
\let\oldabs\abs
\def\abs{\@ifstar{\oldabs}{\oldabs*}}

\let\oldnorm\norm
\def\norm{\@ifstar{\oldnorm}{\oldnorm*}}
\makeatother

\newcommand*{\Value}{\frac{1}{2}x^2}%
\begin{document}

\title{Vectors}	
\author{Edward Jex}
\maketitle

\section*{Idea}
\begin{itemize}
\item (physics) Linear displacement of objects. 
\item (maths) Do geometric algebra.
\item Two vectors are the same if they have the same direction and magnitude. 
\item Vectors represent directed line segments. 
\end{itemize}
\section*{Representation}
\subsection*{Component representation:}
$ \vec{v} = \begin{bmatrix} v_1 \\ v_2 \\ \vdots \\ v_n \end{bmatrix} $\\
\subsection*{Basis representation:}
$ \vec{v} = v_1 \vec{e_1} + v_2 \vec{e_2} + \dots + v_n \vec{e_n} $ \\
In text book $\vec{e_1}, \vec{e_2}, \vec{e_3} $ etc. may be written as $\vec{i}, \vec{j}, \vec{k}$ \\
$\vec{e_n}$ represents a unit vector along one of the axes. E.g. in 2 dimensions, $\vec{e_2} = \begin{bmatrix} 0 \\ 1 \end{bmatrix} $
\subsection*{Polar representation:}
\includegraphics[scale=0.5]{fig1}\\
Where $ \vec{v} = \abs{\vec{v}} \begin{bmatrix} \sin \theta \\ \cos \theta\end{bmatrix} $
\section*{Operations}
\subsection*{Addition}
Vector addition is commutative (order doesn't matter) and associative (bracketing doesn't matter).
\subsection*{Example 1:}
\begin{align*}
\begin{bmatrix} p_1 \\ p_2 \end{bmatrix} + 
\begin{bmatrix} q_1 \\ q_2 \end{bmatrix} &= 
\begin{bmatrix} p_1 + q_1 \\ p_2 + q_2 \end{bmatrix} \\\\
\vec{u} + \vec{w} + \vec{v} & = \vec{u} + ( \vec{w} + \vec{v} ) \\
		  & = \vec{u} + ( \vec{v} + \vec{w} )
\end{align*}
\subsection*{Scalar Multiplication}
$ \lambda \vec{v} \qquad \lambda \in \mathbb{R}$\\
$\lambda$ Scales the vector but keeps the direction if positive, reverses if negative.\\
\subsection*{Example 2:}
\begin{align*}
\lambda \begin{bmatrix} v_1 \\ v_2 \\ \vdots \\ v_n \end{bmatrix} & = 
\begin{bmatrix} \lambda v_1 \\ \lambda v_2 \\ \vdots \\ \lambda v_n \end{bmatrix} \\\\ 
\abs{\lambda \vec{v}} & = \lambda \abs{\vec{v}} \\\\
\abs{\vec{v}} & = \sqrt{v_1^2 + v_2^2 \dots + v_n^2}
\end{align*}
What happens when $\lambda = 0$? We need to define a zero vector: 
\begin{align*}
\vec{o} & = 
\begin{bmatrix}
	0 \\
	0 \\
	\vdots \\
	0 \\
\end{bmatrix}
\end{align*}
Zero vector is the additive identity 

\section*{Vectors and coordinates}
Vector space is any type that the same algebraic rule as a vector (commutativity, associativity and distributivity).
\subsection*{How can we link vectors back to coordinates?}
A vector can represent the displacement from the origin of a point.
$ \vec{v} = \overrightarrow{OP} $ Represents the displacement of point P relative to the origin as we always need a frame of reference. This is called a position vector.
\section*{Applications of vectors}
We can use vectors to represent shapes with position vectors to vertices. From there you can then do calculations with the vectors.
\subsection*{Example 3:}
\includegraphics[scale=0.5]{fig2} \\
\begin{align*}
\overrightarrow{PQ} = \overrightarrow{OP} - \overrightarrow{OP}
\end{align*}
You can also use vectors to represent the equation of a line:
\begin{align*}
v & \in \mathbb{R}^n \\
\vec{x}: \mathbb{R} & \rightarrow \mathbb{R}^n \\
t & \mapsto \vec{v}t + \vec{x_0} \\
\end{align*}
\end{document}