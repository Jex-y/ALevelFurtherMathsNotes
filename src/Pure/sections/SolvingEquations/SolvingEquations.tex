\documentclass[class=article, crop=false]{standalone}
\usepackage[fleqn]{amsmath}
\usepackage{amssymb}
\begin{document}

\section*{Equations with Indices}
If we can write an equation in terms of the same bases, we can equate the powers to help us solve them.
\subsection*{Example 1}
\begin{align*}
&\begin{cases}
	2^{x-y} = 64^2 \\
	3^{x+y} = 1
\end{cases} \\
\text{Find } &  x, y \\
&\begin{cases}
	2^{x-y} = 2^12 \Rightarrow x-y = 12 \\
	3^{x+y} = 3^0 \Rightarrow x+y = 0 \\
\end{cases} \\
2x & = 12 \\
&\begin{cases}
	x = 6 \\
	y = -6 \\
\end{cases} \\
\end{align*}
\section*{Quadratics}
\subsection*{Factorising}
$(x - \alpha)$ is a factor $\iff x = \alpha$ \\
Given an equation of the form $x^2 + bx + c$, the roots should add to b and multiply to c.
\subparagraph*{Difference of two squares}
$x^2 - y^2 = (x+y)(x-y)$
\subparagraph*{Completing the square}
Given an equation $x^2 + bx + c$, complete square form is $(x + \frac{b}{2}) - \frac{b^2}{4} + c$. \\
This can easily be rearranged to get x as the subject. \\
Given the equation $ax^2 + bx +c$, the completed square form can be rearranged to produce the quadratic formula.
\subsubsection{The Quadratic Formula}
$x = \frac{-b \pm \sqrt{b^2 - 4ac}}{2a}$ for an equation $ax^2 + bx + c = 0$.
\subsubsection*{The Discriminant}
Looking at the quadratic formula, you can see that there would be no real solutions if $b^2 -4ac < 0$. \\
$> 0 \iff$ 2 real roots \\
$= 0 \iff$ 1 real root (repeated) \\
\subsection*{Hidden Quadratics}
Polynomials of higher power can be solved like quadratics if they fit the form $ax^{2d} + bx^{d} + c$.
\subsection*{Example 2}
For which values of $k$, does the following equation have repeated roots? \\
\begin{align*}
x^2 + kx + 2k & = 0 \\
k^2 - 8k & = 0 \\
k(k - 8) & = 0 \\
k = 0 \hspace*{1cm} & \hspace*{1cm} k = -8
\end{align*}
\subsection*{Example 3}
\begin{align*}
x^4 - 3x^2 - 4 & = 0 \\
(x^2 - 4)(x^2 + 1)& = 0 \\
x & = \pm 2 \\
\end{align*}
\end{document}