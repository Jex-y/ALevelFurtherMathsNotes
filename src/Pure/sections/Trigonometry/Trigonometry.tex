\documentclass[class=article, crop=false]{standalone}
\usepackage[fleqn]{amsmath}
\usepackage{amssymb}
\usepackage{graphicx}
\graphicspath{ {../images/} }
\begin{document}

\section*{The Radian}
$\pi$ radians $= 180$ degrees \\ 
\subsection*{Arc length}
\begin{align*}
l & = \frac{\theta 2 \pi r}{2 \pi} \\
& = r \theta \\
\end{align*}
\subsection*{Sector area} 
\begin{align*}
A & = \frac{\theta}{2 \pi} \pi r^2 \\
& = \frac{r^2 \theta}{2} \\
\end{align*}
\section*{Trigonometry}
Values of $\sin$, $\cos$ and $\tan$ can be worked out by using triangles.

\begin{center}
\begin{tabular}{ c | c c c c c }
\hline
	 & $0$ & $\frac{\pi}{6}$ & $\frac{\pi}{4}$ & $\frac{\pi}{4}$ & $\frac{\pi}{2}$ \\ 
\hline
	$\sin \theta$ & $0$ & $\frac{1}{2}$ & $\frac{1}{\sqrt{2}}$ & $\frac{\sqrt{3}}{2}$ & $1$ \\ 
	$\cos \theta$ & $1$ & $\frac{\sqrt{3}}{2}$ & $\frac{1}{\sqrt{2}}$ & $\frac{1}{2}$ & $0$ \\ 
	$\tan \theta$ & $0$ & $\frac{1}{\sqrt{3}}$ & $1$ & $\sqrt{3}$ & NaN \\ 
\hline
\end{tabular}
\end{center}
\subsection*{The unit circle}
\subparagraph*{Identities}
\begin{align*}
\tan \theta & \equiv \frac{\sin \theta}{\cos \theta} \\
\sin^2 \theta + \cos^2 \theta & \equiv 1 \\
\end{align*}
\includegraphics[scale=0.7]{UnitCircle} \\\\
$\sin$ and $\cos$ graphs have a period of $2\pi$, $\tan$ has a period of $\pi$. \\
$\sin$ and $\tan$ have a rotational symmetry about the origin. \\
$\cos$ has a line of symmetry on the y axis \\
\begin{align*}
\cos - \theta & = \cos \theta \\
\sin - \theta & = - \sin \theta \\
\tan - \theta & = - \tan \theta \\ 
\end{align*}
\section*{Solving Equations}
Be careful not to divide by an expression that may be 0 as you may lose solutions. Also note that there may be many solutions in a given range. Drawing a CAST diagram or graph sketch may be useful.
\subsection*{Example 1}
\begin{align*}
\text{Solve} \sin \theta - 2 \cos \theta & = 0 \hspace*{1cm} \text{for } 0 \leqslant \theta < 2\pi \\
\sin \theta & = 2 \cos \theta \\
\frac{\sin \theta}{\cos \theta} & = 2 \\
\tan \theta & = 2 \\
\theta & = \arctan 2 \\
\theta & = 1.107, 4.249 \\
\end{align*}
Note, two solutions.
\subsection*{Example 2}
\begin{align*}
\text{Solve } 2 \cos \theta \sin \theta & = \cos \theta \hspace*{1cm} \text{for } 0 \leqslant \theta <  2\pi \\
2 \cos \theta \sin \theta - \cos \theta & = 0 \\
\cos \theta (2\sin \theta - 1) & = 0 \\
\cos \theta = 0 \hspace*{1cm} & \hspace*{1cm} \sin \theta = \frac{1}{2} \\ 
\theta & = \frac{\pi}{6}, \frac{\pi}{4}, \frac{5 \pi}{6}, \frac{3 \pi}{2} \\
\end{align*}
\subsection*{Example 3}
\begin{align*}
\text{Solve } \sin^2 \theta + \sin \theta & = \cos^2 \theta \hspace*{1cm} \text{for } 0 \leqslant \theta < 2 \pi \\
\sin^2 \theta + \sin \theta & = 1 - \sin^2 \theta \\
2 \sin^2 \theta + \sin \theta & = 1 \\
2 \sin^2 \theta + sin \theta - 1 & = 0 \\
(\sin \theta + 1)(2 \sin \theta -1) & = 0 \\
\sin \theta = -1 \hspace*{1cm} & \hspace*{1cm} \sin \theta = \frac{1}{2} \\
\theta = \frac{\pi}{6}, \frac{5 \pi}{6}, \frac{3 \pi}{2} & \\ 
\end{align*}
\section*{Sine and Cosine rules}
\begin{align*}
\frac{\sin A}{a} & = \frac{\sin B}{b} = \frac{\sin C}{c} \\
a^2 &= b^2 + c^2 - 2bc \cos A \\ 
\end{align*}
Area of a triangle $ = \frac{1}{2} ab \sin C$ \\ 
\includegraphics[scale=0.7]{Triangle} \\\\
\section*{Small Angle Approximations}
\includegraphics[scale=0.5]{Graph3} \\
At small angles, trigonometric functions can be approximated. For small $\theta$:
\begin{itemize}
	\item $\sin \theta \simeq \theta$
	\item $\tan \theta \simeq \theta$
	\item $\cos \theta \simeq 1 - \frac{\theta^2}{2}$
\end{itemize}
These approximations can be found in the first few terms of the Taylor series expansions of the trigonometric functions. They can be used to more easily solve functions with small angles. 
\section*{Further Trig Functions}
The reciprocals of trig functions have their own notations. \\
\begin{tabular}{ c c }
\hline
	 $f(x)$ & $\frac{1}{f(x)}$  \\ 
\hline
	$\sin x$ & $\csc x$ \\
	$\cos x$ & $\sec x$ \\
	$\tan x$ & $\cot x$ \\
\hline
\end{tabular}
\\
This means that there are some more identities. 
\begin{align*}
1 + \tan^2 \theta & = \sec^2 \theta \\
\cot^2 \theta + 1 & = \csc^2 \theta \\
\frac{\cos \theta}{\sin \theta} & = \cot \theta \\
\end{align*}
\section*{Compound Angle Formulae}
These are given in the formulae book but it is quicker if you just learn them. 
\begin{align*}
\sin \alpha \pm \beta & = \sin \alpha \cos \beta \pm \sin \beta \cos \alpha \\
\cos \alpha \pm \beta & = \cos \alpha \cos \beta \mp \sin \alpha \sin \beta \\
\tan \alpha \pm \beta & = \frac{\tan \alpha \pm \tan \beta}{1 \mp \tan \alpha \tan \beta} \\
\end{align*}
These are proven via the diagram below. \\
\includegraphics[scale=0.3]{CompoundAngle} \\
\subsection*{Double Angle Formulae}
These come from the compound angle formulae but we just make both the angles the same. 
\begin{align*}
\sin 2 \alpha & = 2 \sin \alpha \cos \alpha \\ 
\cos 2 \alpha & = \cos^2 \alpha - \sin^2 \alpha \\
& = 1 - 2 \sin^2 \alpha \\
& = 2 \cos^2 \alpha - 1 \\ 
\tan 2 \alpha & = \frac{2 \tan \alpha}{1 - \tan^2 \alpha}
\end{align*}
\subsection*{Tripe Angle Formulae}
These can be derived from the compound angle formulae as well. 
\begin{align*}
\sin 3 \alpha & = 3 \sin \alpha - 4 \sin^3 \alpha \\
\cos 3 \alpha & = 4 \cos^3 \alpha - 3 \cos \alpha \\
\end{align*}
\section*{The form $R\cos x \pm \alpha$ and $R \sin x \pm \alpha$}
To make them easier to solve, trigonometric expressions can be written in terms of sin or cosine entirely. 
\subsubsection*{Example 1}
Write $2 \sin x + 3 \cos x$ in the form $R \cos x - \alpha$ where $R > 0$ and $0 < \alpha < 90$ \\
\begin{align*}
R \cos x - \alpha & \equiv R (\cos x \cos alpha + \sin x \sin \alpha) \\
2 \sin x + 3 \cos x & = R \cos x \cos \alpha + R \sin x \sin \alpha \\ 
\Rightarrow & \begin{cases}
	R \sin \alpha = 2 \\
	R \cos \alpha = 3 \\
	\end{cases} \\
\Rightarrow \tan \alpha & = \frac{2}{3} \\
\alpha & = 33.69 \\
\\
\Rightarrow & \begin{cases}
	R^2 \sin^2 \alpha = 4 \\
	R^2 \cos^2 \alpha = 9 \\
	\end{cases} \\
\Rightarrow R^2 (\sin^2 \alpha + \cos^2 \alpha) & = 4 + 9 \\
\Rightarrow R & = \sqrt{13} \\
\therefore 2 \sin x + 3 \cos x & = \sqrt{13}\cos(x-33.69)
\end{align*}
The output is scaled from between $-1$ and $1$ to $-sqrt{13}$ and $sqrt{13}$

There is a short cut to find $R$ as it is just square root of the sum of the two coefficients squared. 
\end{document}