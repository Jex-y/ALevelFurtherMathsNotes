\documentclass[a4paper,12pt]{article}
\usepackage[fleqn]{amsmath}
\usepackage{amssymb}
\begin{document}

\title{Sequences and Series}	
\author{Edward Jex}
\maketitle
A sequence is an ordered list of terms. A series is the sum of a sequence. 
\section*{Notation}
We can write the terms of a sequence as $a_1, a_2, a_3, \dots, a_r$ \\
The sum of the terms $a_1 + a_2 + a_3 + \dots + a_n$ can be written as $\sum_{r=1}^n a_r$ \\

\section*{Summation Using Standard Results}
There are several standard results that can be used to more easily evaluate series. \\
\begin{align*}
\sum_1^n r   & = \frac{(n+1)n}{2} \\
\sum_1^n r^2 & = \frac{n(n+1)(2n+1)}{6} \\
\sum_1^n r^3 & = \frac{(n+1)^2n^2}{4} \\
\end{align*}
\subsubsection*{Example 1}
Find $\sum^n_{r=1}(2r+1)(3r+4)$
\begin{align*}
\sum^n_{r=1}(2r+1)(3r+4) &= \sum^n_{r=1} 6r^2 + 11r + 4 \\
& = 6(\frac{n(n+1)(2n+1)}{6}) + 11(\frac{(n+1)n}{2}) + 4n \\
& = 2n^3 + \frac{17}{2}n^2 + \frac{21}{2} n \\
& = \frac{n}{2}(4n^2 + 17n + 21)
\end{align*}

\section*{Method of Differences}
This can be used to find series without using standard results. 
\subsubsection*{Example 1}
Find $\sum^n_{r=1} r(r+1)$ without quoting standard results. 
\begin{align*}
& r(r+1)(r+2) - (r-1)(r)(r+1) \\
& = r(r+1)((r+2)-(r-1)) \\
& = 3r(r+1) \\
\sum^n_{r=1} 3r(r+1) & = \sum^n_{r=1} r(r+1)(r+2) - (r-1)(r)(r+1) \\
= & \hspace*{5cm} (1 \times 2 \times 3) - (0 \times 1 \times 2) \\
+ & \hspace*{2.5cm} (2 \times 3 \times 4) - (1 \times 2 \times 3) \\
+ & (3 \times 4 \times 5) - (2 \times 3 \times 4) \\
+ & \dots \\
+ & \hspace*{7.2cm} (n-2)(n-1)(n) - (n-3)(n-2)(n-1) \\
+ & \hspace*{3.6cm} (n-1)(n)(n+1) - (n-2)(n-1)(n) \\
+ & (n)(n+1)(n+2) - (n-1)(n)(n+1) \\
= & n(n+1)(n+2) - (0 \times 1 \times 2) \\
= & n(n+1)(n+2) \\
\therefore & \sum^n_{r=1} r(r+1) = \frac{1}{3}n(n+1)(n+2) \\
\end{align*}
\end{document}