\documentclass[a4paper,12pt]{article}
\usepackage[fleqn]{amsmath}	
\usepackage{pgfplots}
\usepackage{amssymb}
\begin{document}

\title{Inequalities}	
\author{Edward Jex}
\maketitle

When multiplying inequalities with negative reals then the relation must swap sign. e.g.: $-x > -4 \iff x < 4$
\\With quadratic inequalities it is always worth sketching a graph so that you can visualise the area that you are trying to get.
\subsection*{Example 1: }
\begin{align*}
x^2 -4x +3 & < 0 \\
(x-3)(x-1) & < 0 \\
\therefore 1 \leqslant x \leqslant 3&\\
\end{align*}
\begin{tikzpicture}
  \begin{axis}[xmin=-2,xmax=6,ymax=10]
  \addplot[blue, samples=100, smooth, domain=-10:10] 
  	plot(\x,{\x^2+(-4*\x)+3}); 
  \addplot[black, domain=-10:10]
  	plot(\x,0);
  \end{axis}
\end{tikzpicture}

\subsection*{Thinking: }
\begin{align*}
ab & > 0 \iff ((a > 0)\;\&\&\;(b > 0))\;||\;((a < 0)\;\&\&\;(b < 0))\\
a,b & \in \mathbb{R}\\
\end{align*}

\end{document}