\documentclass[a4paper,12pt]{article}
\usepackage[fleqn]{amsmath}
\usepackage{amssymb}
\begin{document}

\title{Simultaneous Equations}	
\author{Edward Jex}
\maketitle
Simultaneous equations are set of equations that are all true at the same time. This is often shown using curly braces.
\subsection*{Example 1:}
Note in this example that we have one quadratic and one linear equation. The linear equation should always be substituted in to the quadratic as it makes it much easier to solve. 
\begin{align*}
& 
\begin{cases}
y & = 4 - 2x \\
y & = x^2 + x 
\end{cases}
\\ 
4 - 2x & = x^2 + x 
\\
0 & = x^2 + 3x -4 
\\ 
x & = -4,\;x = 1
\\
\end{align*}
\section*{Solving With Matrices}
Linear simulations equations can be solved by representing them with matrices and vectors. The equations 
$\begin{cases}
a_1 x + b_1 y + c_1 z = d_1 \\
a_2 x + b_2 y + c_2 z = d_2 \\
a_3 x + b_3 y + c_3 z = d_3 \\
\end{cases}$
can be written 
$\begin{bmatrix}
a_1 & b_1 & c_1 \\
a_2 & b_2 & c_2 \\
a_3 & b_3 & c_3 \\
\end{bmatrix}
\begin{bmatrix}
x \\ y \\ z \\
\end{bmatrix} = 
\begin{bmatrix}
d_1 \\ d_2 \\ d_3 \\
\end{bmatrix}$
Therefore, $x$, $y$ and $z$ can be found by rearranging the equation and finding the inverse of the matrix. 
$\begin{bmatrix}
x \\ y \\ z \\
\end{bmatrix} = 
\begin{bmatrix}
a_1 & b_1 & c_1 \\
a_2 & b_2 & c_2 \\
a_3 & b_3 & c_3 \\
\end{bmatrix}^{-1}
\begin{bmatrix}
d_1 \\ d_2 \\ d_3 \\
\end{bmatrix}$

\end{document}