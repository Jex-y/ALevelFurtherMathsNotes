\documentclass[a4paper,12pt]{article}
\usepackage[fleqn]{amsmath}
\usepackage{amssymb}
\usepackage{graphicx}
\graphicspath{ {./images/} }
\begin{document}

\title{Trigonometry}	
\author{Edward Jex}
\maketitle
\section*{The Radian}
$\pi$ radians $= 180$ degrees \\ 
\subsection*{Arc length}
\begin{align*}
l & = \frac{\theta 2 \pi r}{2 \pi} \\
& = r \theta \\
\end{align*}
\subsection*{Sector area} 
\begin{align*}
A & = \frac{\theta}{2 \pi} \pi r^2 \\
& = \frac{r^2 \theta}{2} \\
\end{align*}
\section*{Trigonometry}
Values of $\sin$, $\cos$ and $\tan$ can be worked out by using triangles.

\begin{center}
\begin{tabular}{ c | c c c c c }
\hline
	 & $0$ & $\frac{\pi}{6}$ & $\frac{\pi}{4}$ & $\frac{\pi}{4}$ & $\frac{\pi}{2}$ \\ 
\hline
	$\sin \theta$ & $0$ & $\frac{1}{2}$ & $\frac{1}{\sqrt{2}}$ & $\frac{\sqrt{3}}{2}$ & $1$ \\ 
	$\cos \theta$ & $1$ & $\frac{\sqrt{3}}{2}$ & $\frac{1}{\sqrt{2}}$ & $\frac{1}{2}$ & $0$ \\ 
	$\tan \theta$ & $0$ & $\frac{1}{\sqrt{3}}$ & $1$ & $\sqrt{3}$ & NaN \\ 
\hline
\end{tabular}
\end{center}
\subsection*{The unit circle}
\subparagraph*{Identities}
\begin{align*}
\tan \theta & \equiv \frac{\sin \theta}{\cos \theta} \\
\sin^2 \theta + \cos^2 \theta & \equiv 1 \\
\end{align*}
\includegraphics[scale=0.7]{UnitCircle} \\\\
$\sin$ and $\cos$ graphs have a period of $2\pi$, $\tan$ has a period of $\pi$. \\
$\sin$ and $\tan$ have a rotational symmetry about the origin. \\
$\cos$ has a line of symmetry on the y axis \\
\begin{align*}
\cos - \theta & = \cos \theta \\
\sin - \theta & = - \sin \theta \\
\tan - \theta & = - \tan \theta \\ 
\end{align*}
\section*{Solving Equations}
Be careful not to divide by an expression that may be 0 as you may lose solutions. Also note that there may be many solutions in a given range. Drawing a CAST diagram or graph sketch may be useful.
\subsection*{Example 1}
\begin{align*}
\text{Solve} \sin \theta - 2 \cos \theta & = 0 \hspace*{1cm} \text{for } 0 \leqslant \theta < 2\pi \\
\sin \theta & = 2 \cos \theta \\
\frac{\sin \theta}{\cos \theta} & = 2 \\
\tan \theta & = 2 \\
\theta & = \arctan 2 \\
\theta & = 1.107, 4.249 \\
\end{align*}
Note, two solutions.
\subsection*{Example 2}
\begin{align*}
\text{Solve } 2 \cos \theta \sin \theta & = \cos \theta \hspace*{1cm} \text{for } 0 \leqslant \theta <  2\pi \\
2 \cos \theta \sin \theta - \cos \theta & = 0 \\
\cos \theta (2\sin \theta - 1) & = 0 \\
\cos \theta = 0 \hspace*{1cm} & \hspace*{1cm} \sin \theta = \frac{1}{2} \\ 
\theta & = \frac{\pi}{6}, \frac{\pi}{4}, \frac{5 \pi}{6}, \frac{3 \pi}{2} \\
\end{align*}
\subsection*{Example 3}
\begin{align*}
\text{Solve } \sin^2 \theta + \sin \theta & = \cos^2 \theta \hspace*{1cm} \text{for } 0 \leqslant \theta < 2 \pi \\
\sin^2 \theta + \sin \theta & = 1 - \sin^2 \theta \\
2 \sin^2 \theta + \sin \theta & = 1 \\
2 \sin^2 \theta + sin \theta - 1 & = 0 \\
(\sin \theta + 1)(2 \sin \theta -1) & = 0 \\
\sin \theta = -1 \hspace*{1cm} & \hspace*{1cm} \sin \theta = \frac{1}{2} \\
\theta = \frac{\pi}{6}, \frac{5 \pi}{6}, \frac{3 \pi}{2} & \\ 
\end{align*}
\section*{Sine and Cosine rules}
\begin{align*}
\frac{\sin A}{a} & = \frac{\sin B}{b} = \frac{\sin C}{c} \\
a^2 &= b^2 + c^2 - 2bc \cos A \\ 
\end{align*}
Area of a triangle $ = \frac{1}{2} ab \sin C$ \\ 
\includegraphics[scale=0.7]{Triangle} \\\\
\end{document}