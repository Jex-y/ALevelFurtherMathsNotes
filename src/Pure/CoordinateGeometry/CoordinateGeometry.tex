\documentclass[a4paper,12pt]{article}
\usepackage[fleqn]{amsmath}
\begin{document}

\title{Coordinate Geometry}	
\author{Edward Jex}
\maketitle
To find the midpoint of a line, take the mean of the endpoints.\\\\
Gradient $= \frac{\delta y}{\delta x}$\\\\
For perpendicular lines with gradients $m_1$, $m_2$: $m_1 = \frac{-1}{m_2}$\\\\
General formula of a straight line: $y = mx + c$ or $(y-y_0)=m(x-x_0)$, latter preferred as easier.\\\\
The intersection of lines can be found by solving simultaneously. In 2D this gives 0,1,$\infty$ solutions\\\\
General formula of a circle: $r^2 = (x-a)^2 + (y-b)^2$ where radius is r, center is (a,b)\\\\


\end{document}